\chapter{Общие вопросы}

В этой главе приводятся основные понятия ML и DS. 


\section{Машинное обучение}

Машинное обучение (ML) - область искусственного интеллекта, изучающая самообучающиеся модели, то есть решаюшие поставленную задачу не по заранее запрограммированному алгоритму, а предварительно настраивая свое поведение согласно имеющимся данным. 

Обычно методы ML содержат свободные параметры, подбор которых наилучшим (в смысле имеющихся данных) образом и составляет процесс обучения алгоритма.


\section{Основные классы задач}


\section{Обнаружение аномалий}


\section{Контроль качества}

...оценка обобщающей способности...


\section{Недообучение}


\section{Переобучение}


\section{Регуляризация}


\section{Отбор признаков}


\section{Параметры алгоритма}


\section{Подбора метапараметров}


\section{Основные типы алгоритмов}


\section{Многоклассовая классификация}


\section{Дисбаланс классов}


\section{Ансамбли алгоритмов}


\section{Метрики классификации}


\section{ROC-AUC метрика}


\section{Метрики регрессии}


\section{Метрики кластеризации}


\section{Разложение ошибки алгоритма}


\section{Кривые валидации}


\section{Кривые обучения}


\section{Метрические методы}


\section{Метод ближайших соседей}


\section{Линейные методы}


\section{Линейная регрессия}


\section{Логистическая регрессия}

...отличие от линейной...


\section{SVM}


\section{Ядра и спрямляющие пространства}


\section{Решаюшие деревья}


\section{Случайный лес}

...отличие от беггинга над решающими деревьями...


\section{Градиентный бустинг}


\section{Байесовские методы}



