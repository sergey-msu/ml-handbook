\chapter{Математика}

В этой главе описаны основные математические понятия, необходимые для правильного понимания как основных, так и продвинутых методов ML. Охвачены: теория вероятностей, классическая и байесовская статистика, некоторые вопросы мат. анализа.


\section{Случайная величина}

Случайной величиной (RV) называется числовая функция $X$, определенная на некотором множестве элементарных исходов $\Omega$ (обычно подмножество $\mathbb{R}$ или $\mathbb{R}^n$), 

$$
X: \Omega\rightarrow\mathbb{R}.
$$

С прикладной точки зрения на RV часто смотрят как на генераторы случайных чисел с заданным распределением.

\textbf{Примеры:}
\begin{itemize}
    \item Рост людей, взятых из некоторой группы.
    \item Цвет фиксированного пикселя изображения, взятого из некоторого множества изображений.
    \item Некоторый признак из датасета ML задачи.
\end{itemize}


\section{Распределение случайной величины}

Если RV принимает дискретное множество значений $x_1,x_2,...$, то она полностью определяется значениями их вероятностей: $p_k=\mathbb{P}(X=x_k)$.

Если множество значений RV не дискретно, то RV может быть описана своей функцией распределения (CDF, Cumulative distribution function): $F(x)=\mathbb{P}(X<x)$.

В большинстве прикладных случаев CDF оказывается дифференцируемой функцией. Производная от CDF называется плотностью распределения случайной величины (PDF, Probability density function): $f(x)=F'(x)$. Таким образом, по определению 

$$
\mathbb{P}(a<X<b)=\int_{a}^{b}f(x)dx.
$$


\section{Выборка}

Выборкой называется последовательность RV: $X_1, X_2, ..., X_n$. Предполагается, что все $X_k$ попарно независимы и имеют одно и то же распределение: $X_k \sim X$. В этом случае говорят о выборке из генеральной совокупности $X$. 

На практике выборкой являются конкретные реализации величин $X_k$, то есть последовательность чисел $x_1, x_2, ..., x_n$.

\section{Закон больших чисел}

\section{Статистики}

\section{Bootstrap}

\section{Классический и байесовский подход}
